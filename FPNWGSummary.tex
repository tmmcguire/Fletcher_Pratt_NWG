\documentclass[twocolumn]{article}

\usepackage{bookman}
\usepackage[T1]{fontenc}
\usepackage[pdfusetitle]{hyperref}
% \usepackage{multicol}
\usepackage{setspace}
\usepackage{siunitx}
\usepackage{titlesec}

\titleformat{\section}
  {\large\bf}
  {Rule \thesection: }
  {0em}{}
\renewcommand*\sectionautorefname{Rule}

\onehalfspacing

\begin{document}

\title{{\small A Brief Summary of} \\ Fletcher Pratt's Naval Wargame}
\author{Tommy M. McGuire}
\date{13 Jul 2019}

\maketitle

Being a short introduction to the best rules for a wargame of naval miniatures ever created.

% \begin{multicols}{2}

\section*{Introduction}

\section{The apparatus}

\begin{itemize}
\item \textbf{Players.}
  No game is complete without one or more players.
\item \textbf{Referees.}
  Yes, the Fletcher Pratt game does need referees. If none are
  available, other players can act as referees. For example, one player
  fires their ship's guns and another player resolves the results.
\item \textbf{Ship models.}
  The original scale used for ships was 1:600 (yes, 1:600; \emph{Bismark}
  would be about 16'' or 42cm long); 1:1200 is pretty reasonable and I
  would suspect 1:2400 is probably the limit. 1:6000 may cause issues
  with firing (See \autoref{rule3shooting}).
\item \textbf{Playing surface.}
  This is a naval game: any largish, reasonably flat surface will
  likely work. For ``largish'', consider that the initial salvos of
  the Battle of the Denmark Strait took place at 26,500~yards or
  13$\frac{1}{4}$~nautical miles. The large games during WWII were
  held in a ballroom, using a scale of approximately 1:3600. In this
  scale, the Denmark Strait initial range would be a little over 22',
  265'', or 670cm. Pratt's original games took place on his apartment
  floor, using a scale of approximately 1:9000; that is about 9', 106'',
  or 270cm. Donald Featherstone's variant uses a scale of 1:18,000:
  4$\frac{1}{2}$', 53'', or 135cm. In any case, the game scale will need
  to be larger than that of the ship models, but not so large as to make
  the ship spacing look ridiculous. In this summary, I'm going to use
  full-scale measurements with translations of important measures
  into those three scales, which I will term ``large'', ``medium'', and
  ``small''.
\item \textbf{Ship cards.}
  Each ship a paper representation of its capabilities as well as
  ability to take damage. See the details below.
\item \textbf{Firing arrows.}
  The Fletcher Pratt Wargame involves no randomness. Instead, firing
  is handled by the player laying out a marker showing the direction
  of fire and including the estimated range and which of the player's
  ship's guns is firing. See \autoref{rule3shooting} for the details,
  but a supply of paper markers is necessary for each player.

  Pratt's original game used tacks or pins to secure the firing arrows
  to the surface (that would be the ballroom floor). You may not want to
  do the same in your games. However, the firing arrows need to be secured to the
  surface by the players to supply the direction of aim for the referees.
\item \textbf{Miscellaneous.}
  \emph{A stopwatch or clock:} Some of the phases in turns are timed, to
  maintain the flow of the game as well as to emphasize the stress of
  battle.
  \emph{Distance measuring devices:} such as tape measures. Yes, this
  is a miniatures game; players will need to measure movement distances
  and referees will need to measure actual firing distances.
  \emph{Shell splash markers:} Something to indicate where a ship's
  volley landed. One option is upturned golf tees, white for misses and
  red for hits.
\end{itemize}

\subsection*{The ship card}

Each ship needs a record containing the following information:
\begin{itemize}
\item The ship's displacement.
\item Number and size of all guns in the ship's primary, secondary, and tertiary batteries as well as the number of anti-aircraft guns, torpedo tubes, and aircraft carried.
\item Belt, deck, and turret armor depth.
\item Maximum speed.
\item Point value.
\item The schedule of damage and its effects on the ship.
\end{itemize}

The point value is calculated using the ship's armaments, armor,
speed, and displacement. It is used both to balance scenarios and to
calculate the damage schedule. For the latter, the point value is
divided by speed of the ship to determine how incoming shells cause
propulsion damage. Likewise, the point value is divided by the number of
primary guns to determine when guns are taken out of action. See
\autoref{laargentinadamageschedule} for an example.

\begin{table*}
  \caption{\emph{La Argentina} (23,034 points) damage schedule.}
  \label{laargentinadamageschedule}
  \centering
  \begin{tabular}{rl|rl|rl}
   \textbf{Damage} & \textbf{Remaining} & \textbf{Damage} & \textbf{Remaining} & \textbf{Damage} & \textbf{Remaining} \\
      0 & 31 kn      & 11145 & 14 kn           & ...   & ...  \\
    743 & 30 kn      & 11516 & 2 4'' guns      & 19575 & 4 kn \\
   1486 & 29 kn      & 11517 & 3 torpedo tubes & 20318 & 3 kn \\
   2229 & 28 kn      & 11888 & 13 kn           & 20472 & 1 6'' gun \\
   2259 & 8 6'' guns & 12631 & 12 kn           & 21061 & 2 kn \\
   2972 & 27 kn      & 12795 & 4 6'' guns      & 21804 & 1 kn \\
   ...  & ...        & 13374 & 11 kn           & 23034 & wrecked
  \end{tabular}
\end{table*}

Ship cards and a spreadsheet for creating them are available from the web page for the book, 
\href{http://www.wargaming.co/recreation/details/fpnaval.htm}
  {\emph{Fletcher Pratt's Naval Wargame Wargaming with model ships 1900-1945}},
at the History of Wargaming Project.

\section{Moving}

One knot is 1 nautical mile, or 2000 yards, per hour. Each turn is
approximately 3$\frac{3}{4}$ minutes. As a result, each ship can
move a maximum of 125 scale yards per turn, per knot of speed it has
remaining. (1$\frac{1}{4}$'' at large scale, $\frac{1}{2}$'' at medium,
and $\frac{1}{4}$'' at small. (Call it 3cm, 1cm, and 5mm in Roman
Catholic.))

For example, \emph{La Argentina}, capable of 31 knots, moves at most
3,875 yards per turn. At large scale, that is $38\frac{3}{4}$'' or 93cm,
$15\frac{1}{2}$'' or 31cm at medium, and $7\frac{3}{4}$'' or 15.5cm at
small.

At a minimum, a ship that can do so must move at least 2 knots.

Each ship can turn up to \ang{45} before moving and again up to \ang{45} after
moving.

\section{Shooting}
\label{rule3shooting}

Gunfire is handled through the combination of player skill and referee measurement.

\subsection*{Player skill}

Each player firing lays a firing arrow at the bow or stern of their
ship, \emph{accurately} pointed towards the target, and records the
player's estimate of the range to the target (in scale measurements)
and the number and size of guns firing. The penalties for doing so
incorrectly are:
\begin{itemize}
\item \textbf{Firing beyond the gun's extreme range:} The shot does not
count and the firing ship takes damage equivalent to one hit from the
firing gun. For gun ranges, see \autoref{gunnerydamageandranges}.
\item \textbf{Firing guns that do not bear:} Shots that are masked by
the ship's superstructure (firing a stern-facing turret at a target off
the ship's bow, for example) do not count.
\item \textbf{Firing \emph{through} another ship:} The shot does not
count and the firing ship takes damage equivalent to one hit from the
firing gun. A large ship such as a battleship or cruiser \emph{can} fire
over a small ship such as a destroyer.
\end{itemize}
(This is where 1:6000 ship models have a problem, by the way. They are
too dang small to aim firing arrows at precisely.)

\subsection*{Referee measurement}

\begin{table*}
  \caption{\emph{La Argentina} 9 gun broadside at 12,000 yards. Scale ranges heavily rounded.}
  \label{laargentinabroadside}
  \centering
  \begin{tabular}{rl|ll|ll|ll}
  & & \multicolumn{6}{c}{\textbf{Scale}} \\
  \textbf{Impacts} &
  \textbf{Distance} &
  \multicolumn{2}{c}{\textbf{Large}} &
  \multicolumn{2}{c}{\textbf{Medium}} &
  \multicolumn{2}{c}{\textbf{Small}} \\
  2 shells & 12,000 yd & 120'' & 305cm & 48''   & 122cm & 24''    & 61cm \\
  1 shell  & 11,900 yd & 119'' & 302cm & 47.5'' & 121cm & 23.75'' & 60.5cm \\
  1 shell  & 12,100 yd & 121'' & 308cm & 48.5'' & 123cm & 24.25'' & 61.5cm \\
  1 shell  & 11,800 yd & 118'' & 299cm & 47''   & 120cm & 23.5''  & 60cm \\
  1 shell  & 12,200 yd & 122'' & 311cm & 49''   & 124cm & 24.5''  & 62cm \\
  1 shell  & 11,700 yd & 117'' & 296cm & 46.5'' & 119cm & 23.25'' & 59.5cm \\
  1 shell  & 12,300 yd & 123'' & 314cm & 49.5'' & 125cm & 24.75'' & 62.5cm \\
  1 shell  & 11,600 yd & 116'' & 293cm & 46''   & 118cm & 23''    & 59cm \\
  \end{tabular}
\end{table*}

The referee extends a straight line from the end of the firing arrow
closest to the ship, along the line of the arrow, for the distance of
the player's estimated range. This gives the center of the salvo's
impact. Pratt used two pins (or tacks) to anchor the firing arrows, and
ran the line along the two pins.

Two shots from the salvo fall at this center point. The next shot falls
100 yards closer to the firer, the next 100 yards farther from the
center point, then 200 yards closer, then 200 yards farther away, and so on.

For example, suppose \emph{La Argentina} fires a broadside of its 9 6''
guns at a target 12,000 yards away. Two shells land at 12,000 yards, one
shell lands at 11,900 yards, one at 12,100 yards, one at 11,800 yards,
one at 12,200 yards, 11,700 yards, 12,300 yards, and 11,600 yards. See
\autoref{laargentinabroadside} for the distances in scale terms. (At
1:18,000 scale, you may want to replace upturned golf tees with map pins
or something.)

Hits are recognized as all shots landing on the ship as well as shots
not more than 100 yards beyond it. Shots falling short do not hit. For
damage resolution, see \autoref{rule4armor}.

\section{Armor}
\label{rule4armor}

\begin{table*}
  \caption{1900--1945 Gunnery: Damage and ranges. See \autoref{scaledistanceconversions} for the factors to convert yards to various scales.}
  \label{gunnerydamageandranges}
  \centering
  \begin{minipage}{6in}
  \begin{tabular}{lrrr|lrrr}
               &                 & \multicolumn{2}{c}{\textbf{Range (yards)}} &
               &                 & \multicolumn{2}{c}{\textbf{Range (yards)}} \\
  \textbf{Gun} & \textbf{Damage} & \textbf{Old guns\footnote{WWI-era}} & \textbf{New guns\footnote{WWII-era}} &
  \textbf{Gun} & \textbf{Damage} & \textbf{Old guns} & \textbf{New guns} \\
  37mm         &     23 & 7500   & 7500         &  5.9in        &    490 & 25,000    & 34,000       \\
  40mm         &     26 & 7500   & 7500         &  6in          &    500 & 25,000    & 34,000       \\
  47mm         &     38 & 7500   & 7500         &  6.1in        &    550 &           & 34,500       \\
  57mm         &     45 & 7500   & 7500         &  7.5in        &    880 & 32,500    & 37,500       \\
  2.6in        &     60 & 7500   & 7500         &  7.6in        &    940 & 32,500    &              \\
  3in          &     90 & 8750   & 10,000       &  8in          &   1150 & 32,500    & 40,000       \\
  3.1in        &    100 &        & 12,500       &  8.2in        &   1300 & 34,000    &              \\
  3.4in        &    105 & 10,000 & 12,500       &  9.2in        &   1500 & 34,000    &              \\
  3.5in        &    110 &        & 12,500       &  9.4in        &   1650 & 35,000    &              \\
  3.9in        &    152 & 11,250 & 17,500       &  10in         &   1960 & 35,000    & 40,000       \\
  4in          &    160 & 12,500 & 17,500       &  11in         &   2600 & 37,500    & 41,000       \\
  4.1in        &    192 & 15,000 & 20,000       &  12in         &   3700 & 32,500    & 41,000       \\
  4.5in        &    212 &        & 25,000       &  12.6in       &   4700 &           & 41,500       \\
  4.7in        &    244 & 17,500 & 31,500       &  13in         &   5150 & 34,000    & 42,000       \\
  5in          &    266 & 18,750 & 31,500       &  13.4in       &   5200 &           & 41,500       \\
  5.1in        &    352 &        & 32,000       &  13.5in       &   5300 &           & 41,250       \\
  5.25in       &    372 &        & 32,000       &  14in         &   6250 &           & 42,000       \\
  5.3in        &    390 &        & 32,000       &  15in         &   8550 &           & 42,000       \\
  5.5in        &    404 & 20,000 & 32,500       &  16in         & 10,550 &           & 42,500       \\
               &        &        &              &  18in         & 13,000 &           & 42,500  
  \end{tabular}
  \end{minipage}
\end{table*}

\begin{table*}
  \caption{Scale distance conversions: yards to scale inches/centimeters.}
  \label{scaledistanceconversions}
  \centering
  \begin{tabular}{cc|cc|cc}
  \multicolumn{2}{c}{\textbf{Large}} & \multicolumn{2}{c}{\textbf{Medium}} & \multicolumn{2}{c}{\textbf{Small}} \\
  \textbf{in} & \textbf{cm} & \textbf{in} & \textbf{cm} & \textbf{in} & \textbf{cm} \\
  $\div 100$  & $\div 40$   & $\div 250$  & $\div 100$   & $\div 500$   & $\div 200$ 
  \end{tabular}
\end{table*}

The first step in resolving damage is to determine if the shot has penetrated the relevant armor.
\begin{itemize}
\item Hits directly falling on turrets are compared to turret armor.
In addition, if the hit penetrates the armor then the turret is out of
action, reducing the number of guns available in addition to normal
damage computation.
\item If the ship is broadside to the direction of fire, hits are
compared to belt armor.
\item If the ship is bow- or stern-on to the direction of fire, hits are
compared to deck armor.
\end{itemize}

\section*{The rest: Torpedoes, submarines, airplanes, smoke screens, and mines}

The rules used by Fletcher Pratt during WWII handled torpedoes, smoke
screens, and mines very well; airplanes less so; and submarines somewhat
poorly. For simplicity of this summary (and to encourage you to buy the
book from
\href{http://www.wargaming.co/recreation/details/fpnaval.htm}{The History of Wargaming Project},
I have not included the rules for those in this summary.

% \end{multicols}

\end{document}
